%!TEX program = xelatex
\documentclass[table,aspectratio=169]{beamer}

%\usefonttheme{professionalfonts}
\usepackage[T1]{fontenc}
\renewcommand*\familydefault{\sfdefault}

\usepackage{beamerthemesplit}
\usepackage{graphicx}
\graphicspath{{./Figures/}}

\usepackage{bm,bbm,bbding}
\usepackage{natbib}
\usepackage{siunitx}
\usepackage{texnames}
\usepackage{algorithm,algpseudocode}
\usepackage{multirow}
\usepackage{booktabs}
\usepackage{amsmath}

\definecolor{anti-flashwhite}{rgb}{0.95, 0.95, 0.96}
\definecolor{byzantine}{rgb}{0.74, 0.2, 0.64}
\definecolor{VUWlightgreen}{rgb}{0, .294, .204}
\definecolor{VUWgreen}{rgb}{0, .196, .136}

\usetheme[numbering=fraction]{metropolis}

\setbeamercovered{dynamic}
\setbeamercolor{background canvas}{bg=VUWgreen}
\setbeamercolor{normal text}{fg=anti-flashwhite, bg=VUWgreen}
\setbeamercolor{frametitle}{bg=VUWlightgreen,fg=anti-flashwhite}
\setbeamercolor{footline}{bg=VUWlightgreen}

\makeatletter
\setlength{\metropolis@frametitle@padding}{.5em}% <- default 2.2 ex

\setbeamertemplate{footline}{%
	\begin{beamercolorbox}[wd=\textwidth, sep=.2ex]{footline}% <- default 3ex
		\usebeamerfont{page number in head/foot}%
		\usebeamertemplate*{frame footer}
		\hfill%
		\usebeamertemplate*{frame numbering}
	\end{beamercolorbox}%
}
\makeatother

\setbeamertemplate{frame footer}{\includegraphics[width=1.5cm]{"Logo Offshore Reversed Landscape RGB"}}
\setbeamertemplate{navigation symbols}{}

\AtBeginSubsection[]
{
	\begin{frame}
		\frametitle{Table of Contents}
		\tableofcontents[currentsection,currentsubsection]
	\end{frame}
}

\title{Identifying Heterogeneity in SAR Data with New Test Statistics}
\subtitle{There are no too old problems}
\author{Alejandro C.\ Frery}
\institute{Victoria University of Wellington\\
	School of Mathematics and Statistics\\
	New Zealand}
\date{November, 2024}

\begin{document}
	
	\setsansfont[BoldFont={Avenir Heavy}]{Avenir Book}
	
	\frame{\titlepage}
	
	\begin{frame}
		This is a joint work with Prof.\ Abraão Nascimento and MSc.\ Janeth Alpala (it is part of her PhD project), from Universidade Federal de Pernambuco, Brazil.
	\end{frame}
	
	\begin{frame}{Starting point}
		\begin{itemize}[<+->]
			\item Synthetic Aperture Radar (SAR) technology is essential for environmental monitoring and disaster management. 
			\item It provides
			images under various conditions, including day or night and adverse weather
			situations~\citep{Moreira2013,Mu2019}. 
			\item The effective use of SAR
			data relies on understanding of their statistical properties
			because it is corrupted by speckle: a noise-like interference effect~\citep{Argenti2013}.
		\end{itemize}
	\end{frame}
	
	\begin{frame}[standout]
		\frametitle{Takeaways}
		You may forget about the technical contents of this talk, but remember:
		\begin{itemize}[<+->]
			\item There are no ``too old'' problems that cannot be revisited.
			\item The data statistical properties always bring valuable information.
			\item A careful mathematical manipulation often leads to new exciting results.
		\end{itemize}
	\end{frame}
	
	
	\begin{frame}[allowframebreaks,allowdisplaybreaks]
		\bibliographystyle{agsm}
		\bibliography{references} 	
	\end{frame}
	
	\begin{frame}[standout]
		\frametitle{The End}
		\begin{columns}
			\begin{column}{.4\linewidth}
				\centering
				\includegraphics[width=.7\linewidth]{QRCode-WebPageVUW}\\	\includegraphics[width=\linewidth]{"Logo Offshore Reversed Landscape RGB"}
			\end{column}
			\begin{column}{.6\linewidth}
				Thanks for your kind attention!
				
				Questions?
			\end{column}
		\end{columns}
		
	\end{frame}
	
\end{document}
