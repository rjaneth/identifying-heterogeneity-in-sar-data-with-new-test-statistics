\documentclass[11pt]{report}

\usepackage[pdftex]{geometry}
\geometry{a4paper,left=2.8cm,right=2.8cm,top=1.5cm,bottom=1.5cm,twoside}

\usepackage{amssymb}
\usepackage{enumerate}
\usepackage{graphicx}
	\graphicspath{./Comments/}
\usepackage[many]{tcolorbox}
\usepackage{natbib}
\usepackage{hyperref}
\usepackage{subfigure}
\usepackage{bm,bbm}
\usepackage{float}

\usepackage[utf8]{inputenc}
\usepackage[T1]{fontenc}

\newtcolorbox{reviewbox}[1]{
colback = white!5!white,
colframe = purple!75!black,
fonttitle=\bfseries,
title=#1
}

\newtcolorbox{responsebox}[1]{
	colback = white!5!white,
	colframe = white!75!black,
	fonttitle=\bfseries,
	title=#1
}

%%%%%%%%%%%%%%%%%%%%%%%%%%%%
\begin{document}


\begin{center}
\large{\textbf{Response Letter to Reviewer \# 1}}

\vglue 0.3cm

\huge{ Identifying Heterogeneity in SAR Data with New Test Statistics\\ (remotesensing-3021303 )}
\end{center}

%\authors
\begin{center}
\textbf{Alejandro C.\ Frery, Janeth Alpala, Abraao D.\ C.\ Nascimento }
\end{center}

\date{\today}


%%%%%%%%%%%%%%%%%%%%%%%%%%%%

\vspace{2cm}
\noindent Dear Editors
\bigskip

\noindent We are grateful to the reviewers for the valuable comments and the time spent on checking our manuscript. 
We have addressed their comments towards improving the paper contents and presentation. 
Below we detail the changes and updates incorporated into the manuscript in this round of review.

\medskip


%-------------------------------------
\begin{reviewbox}{Comment 1}
The paper estimates homogeneous areas, and conversely heterogeneous areas, in SAR imagery via speckle statistics by exploiting variations in scattering at the pixel level. Teasing out information from this statistical behavior remains an intricate, delicate process; one which you appear to have accomplished. 

I have only a few comments and suggestions for clarification that, hopefully, will enhance the readability \& ultimate applicability of the ideas presented.

1. The 3rd to the last paragraph in the Introduction seems redundant. First entropy is discussed and then the 2 CV estimates, so why return again to entropy? (Perhaps some of this paragraph could be incorporated into the previous entropy discussion.)
\end{reviewbox}

\begin{responsebox}{Response}
We agree.
We have merged the third-to-last paragraph with the one that discusses the Shannon entropy.
The new paragraph is
\begin{quote}
	\textcolor{blue}{Entropy is a fundamental concept in information theory with far-reaching applications in pattern recognition, statistical physics, image processing, edge detection and SAR image analysis~\cite{Presse2013,MohammadDjafari2015,Avval2021, Nascimento2014,Nascimento2019}. 
		Shannon introduced it in 1948~\cite{Shannon1948} for a random variable to measure information and uncertainty.
		Shannon entropy is a crucial descriptive parameter in statistics, especially for evaluating data dispersion and performing tests for normality, exponentiality and uniformity~\cite{Wieczorkowski1999,Zamanzade2012}. 
		Entropy estimation is challenging, especially when the model is unknown;
		in these cases, non-parametric methods, as those based on spacings~\cite{AlizadehNoughabi2010,Subhash2021}, can be used.
		This strategy is flexible and robust because it does not enforce a model or parametric constraints.
		Different roughness levels, which materialize as different models for SAR data, have different entropy values.
		We designed a bootstrap-improved non-parametric estimator for the Shannon entropy.}
\end{quote}
Please note that the references are correctly numbered and formatted in the revised manuscript.
\end{responsebox}

\begin{reviewbox}{Comment 2}
2. In Eq. (8), the limit of alpha -> -inf is not transparent. The first 5 terms match eq. (7), and the ln(gamma) \& digamma terms seem to cancel, however, that leaves the ln(-1-alpha) \& L factors. A short discussion, or footnote, would help the reader. 
\end{reviewbox}

\begin{responsebox}{Response}
	content...
\end{responsebox}

\begin{reviewbox}{Comment 3}
3. Fig. 4 Q-Q plots: Are the dashed green lines fits or theoretical estimates of the distributions? What do the different scales imply? (Perhaps a larger variance in one of the distributions?) A short note in the caption or clarification in the text would help. (The later Q-Q plots all show an x=y line for reference. Why the change here?)
\end{reviewbox}

\begin{reviewbox}{Comment 4}
4. What does 'nominal levels of 1\%, 5\%, and 10\%' mean? And, how are these 'achieved' in all cases? This paragraph is the only place where 'nominal levels' are mentioned, please clarify their meaning \& significance.
\end{reviewbox}

\begin{reviewbox}{Comment 5}
The mathematics is, by necessity, a bit dense, therefore I've concentrated my comments on the flow of the paper so that the casual reader may readily follow. To fully follow, comprehend \& replicate this work will require effort on the reader's part, nonetheless clarification of the above points will assist in understanding the gist of the paper. 
\end{reviewbox}

\bibliographystyle{plain}
\bibliography{../references}

\end{document}

